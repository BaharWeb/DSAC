\hypertarget{sec:conclusion}{%
\chapter{Conclusion and Outlook}\label{sec:conclusion}}

Concluding the thesis, this section summarizes the results and contributions, reflects on problems that were not addressed or require further investigation and points out open research challenges.

\hypertarget{thesis-summary}{%
\section{Thesis Summary}\label{thesis-summary}}

Initially motivated by the observed difficulties of several ISVs to commence web migration in the context of industrial research collaboration projects, this thesis systematically investigated the problem domain.
It outlined the reasons for web migration on the one hand and the complexity of the initial legacy situation on the other hand, identifying effort and risk as the two main obstacles.
We detailed the situation of medatixx as representative sample of capable modern medium-sized ISV struggling to bring its non-web legacy systems to the web, analyzing characteristics of company, development, and legacy systems as well as migration objectives.
Based on this, requirements for an appropriate solution have been elicited.
Analysis of the state of the art in the context of a systematic mapping study revealed a lack of approaches supporting the initial phase of web migration to address concerns about effort and risk that are tailored to the characteristics of SME-sized ISVs, despite academia having shifted focus away from web migration.
Based on field research and the application of LFA and HCD methods, a research and solution design was derived which defines research objectives to address ISV's doubts about feasibility and desirability.

The proposed solution is AWSM, which provides a methodology for supporting ISVs to commence web migration.
AWSM specifies principles, formalisms, methods and tools which are complementary to existing comprehensive web migration approaches and address their identified shortcomings.
The AWSM principles base it on open web standards, shape it as methodology for integration with web migration approaches on different degrees of model-driven adoption and advocate the use of rapid prototyping.
The AWSM formalisms provide the conceptual basis ensuring interoperability through mathematical modeling and consistent mapping to the KDM OMG standard.
The three AWSM methods each address a shortcoming of existing approaches that contributes to the doubts about feasibility and desirability:

\begin{itemize}
\tightlist
\item
  AWSM:RE allows to identify and maintain existing valuable knowledge through crowdsourced concept assignment supported by a web-based annotation platform.
\item
  AWSM:RM minimizes risk through migration pilots and demonstrates desirability and feasibility of a potential web-based version of the legacy system applying the rapid prototyping paradigm to web migration.
\item
  AWSM:CI allows to control the impact of web migration on customers through measuring visible changes in the user interface.
\end{itemize}

The AWSM Platform comprises tools that support these methods and the overall AWSM approach through automated or semi-automated transformations, analysis, integration with existing development and management software and providing queryable standards-based representations.
In particular, the AWSM Strategy Selection Decision Support System facilitates ISV's specification of an overall web migration strategy through faceted scenario-based search based on the data from our comprehensive systematic mapping study comprising 122 published web migration approaches and software tools.

The AWSM Reverse Engineering method facilitates recovery of problem and solution domain knowledge from the legacy codebase for ISVs with limited resources through a novel crowdsourced concept assignment strategy.
AWSM:RE integrates with ongoing development as well as with other web migration methods leveraging a queryable open web standards-based knowledge representation.
The AWSM Risk Management method facilitates the demonstration of feasibility and desirability of web migration and the plausibility of a web-based version of the legacy system through a novel rapid web migration prototyping strategy.
The AWSM Platform supports this strategy with a WebAssembly-based toolchain and a guidance and automation system.
The AWSM Customer Impact method facilitates the measurement and control of visible change in user interfaces resulting from web migration through a novel UI similarity measurement strategy.
The measurements can be calibrated to the target user group and are supported through a computer-vision based UI element detection and computation of UI layout distances from visual analysis.

Effectiveness and applicability of the AWSM Methodology and Platform has been evaluated in five experiments combining empirical and objective data.
The experiments demonstrated the feasibility of the proposed techniques with limited resources and limited web engineering and migration expertise, as well as the quality of the achievable results.
The support tools of the AWSM Platform were consistently found useful by the test subjects.
The main identified problems were task complexity, addressed by an additional guidance and automation system, and the non-mature nature of understanding perceptual similarity of user interfaces, which is still a young field with ongoing research.
Analysis of the the AWSM Methodology and Platform in the context of stated requirements confirmed that the original objectives have been achieved.

\hypertarget{lessons-learned}{%
\section{Lessons Learned}\label{lessons-learned}}

The research conducted for this PhD thesis yielded findings that are not specific to the AWSM Methodology and Platform but relevant for research in the web migration field and related areas in general:

\begin{itemize}
\tightlist
\item
  In academia, the interest in core web migration topics has significantly dropped, with the majority of published approaches being of limited relevance due to outdated target web technologies.
Recently, the research focus shifted towards more modern technologies like Cloud.
However, many ISVs still struggle with basic web migration and cannot target the Cloud.
The lack of research interest and resulting unavailability of solutions intensifies this problem.
\item
  A reconsideration of web migration in the light of new technologies like WebAssembly can be fruitful, as they allow different migration paths due to opening up the formerly restricted technological environment of the web towards various source technologies.
\item
  The transfer of paradigms that have been successful in forward software and web engineering to web migration, such as Crowdsourcing and Rapid Prototyping can provide new impetus.
\item
  Despite fake contributions even in a limited group of crowdworkers, suitable quality control measures can effectively ensure results quality and some crowdworkers exhibit an unexpected degree of active commitment and investment of time to provide good results quality.
\item
  The effort for complex web migration activities can be reduced through improved process guidance, partial automation and heuristic suggestions which is measurable in both required time and empirical evaluation.
\item
  Perceptual similarity of user interfaces is important for web migration and influence factors like Order, Orientation and Density constitute a very basic, applicable model, but research in this field is still very young and ongoing.
\item
  While a high degree of integration of web migration activities with ongoing agile development is a crucial feasibility factor for ISVs, it has to be acknowledged that full integration is an ideal that cannot be achieved.
As a requirement for web migration, however, it is valuable since it has not been considered in research yet.
\end{itemize}

\hypertarget{contributions}{%
\section{Contributions}\label{contributions}}

This thesis produced models, methods, techniques and architectures that support ISVs to commence web migration.
The following list summarizes the research contributions of the thesis:

\begin{itemize}
\tightlist
\item
  Specification of a dedicated web migration initiation methodology for SME-sized ISVs with non-web legacy systems and large user bases.
\item
  Systematic problem analysis of the main stakeholder situation and identification of main research objectives through a field research, HCD and LFA-based research process.
\item
  Systematic mapping study of both academic publications and software tools in the web migration field with focus on the SME perspective.
\item
  Development of a Decision Support System facilitating Web migration strategy selection for SMEs through scenario-based, faceted search based on the systematic mapping data.
\item
  Definition of a role and process model for reverse engineering through concept assignment integrated into ongoing development and specification of a support platform architecture integrated with ISV's software infrastructure.
\item
  Definition of a queryable open web standards-based representation of knowledge in legacy codebases through ontological modeling and specification of a storage and querying knowledgebase architecture.
\item
  Specification of a novel Crowdsourced Reverse Engineering strategy through transfer of the Crowdsourcing paradigm to the Reverse Engineering domain by re-formulation of concept assignment as classification problem.
\item
  Specification of a novel Rapid Web Migration Prototyping strategy through transfer of the Rapid Prototyping paradigm to the Web Migration domain by leveraging current open web standards.
\item
  Specification of a novel calibratable UI similarity measurement strategy through modeling computable similarity measures by calculation of visual layout distances between non-web and web user interfaces.
\item
  Insights on perceptual similarity and its objective and subjective impact factors in the context of ongoing research on visual UI similarity perception.
\item
  Systematic empirical experimentation with the proposed methods and tools combining both subjective and objective measures for analysis of effectiveness, applicability and results quality.
\end{itemize}

\hypertarget{ongoing-and-future-work}{%
\section{Ongoing and Future Work}\label{ongoing-and-future-work}}

The AWSM Methodology and Platform addressed three crucial challenges related to initiation of web migration for ISVs.
The proposed methods and tools aimed at lowering the initial barrier originating in effort and risk related with web migration by dedicatedly addressing doubts about feasibility and desirability.
Legacy knowledge recovery, demonstration of feasibility, desirability and plausibility of a web-based version of the legacy system and customer impact control through UI similarity measurements are enabled by various AWSM techniques.
The tools of the AWSM Platform enable ISVs with limited resources and limited web engineering and web migration expertise to successfully apply these techniques.
Future work should focus on enhancing the efficiency and scope of the proposed mechanisms, explore other issues under the same motivation that are not addressed by AWSM and further investigate the perception of user interfaces.

\hypertarget{methodology-and-platform-improvements}{%
\subsection{Methodology and Platform Improvements}\label{methodology-and-platform-improvements}}

The AWSM Methodology and Platform can be improved with regard to its functional scope and efficiency in several ways.
The AWSM Reverse Engineering method based on concept assignment specifies a generic and queryable representation of knowledge \(k=(t,r)\) (cf.~\cref{eq:knowledge}) and its location \(l=(s,f)\) (cf.~\cref{eq:location}) in the codebase facilitated by the SCKM Ontology and SPARQL Endpoint.
Due to principle P2 and P3 no restrictions or further specifications are made for the internal representation \(r\).
Increasing the scope of AWSM:RE would allow to put more emphasis on the extraction process from extension \autocite{Chen2010FeatureLocation} \(l\) to intension \autocite{Chen2010FeatureLocation} \(k\), and a more detailed specification of \(r\).
This can be achieved in different ways: Combining AWSM:RE concept assignment with automatic reverse engineering methods can make use of the knowledge type information \(t\) to run dedicated knowledge extractors for the specific knowledge type on the related extensions, benefiting from AWSM:RE concept assignment similar to the classifiers running on previously detected ROIs in \cref{sec:segmentation.impl}.
Integration with comprehensive web migration approaches as described in \cref{sec:re.conceptual.integration} specifies \(r\) according to the specific models required, e.g.~UWA models, and adding model-specific extraction processes.
The third option is to extend our experiments on Crowdsourced Reverse Engineering towards the extraction of specific problem and solution domain knowledge representations, e.g.~UML diagrams of persistence models or BPMN diagrams of business processes, by the crowd, for instance through microtasking with a more comprehensive classification ontology specific to \(\mathfrak{L}\).
This requires more research to transfer the benefits observed in CSRE and adapt the quality control measures to the new activity type.

Within CSRE, balancing controlled disclosure with readability is a challenge.
\Cref{alg:anonym} presents a simple anonymization technique but AWSM:RE would benefit form a more sophisticated algorithm that improves crowdworkers' code comprehension while maintaining the three anonymization properties defined in \cref{sec:csre.impl}.
For Crowdsourced Reverse Engineering, quality control is crucial.
We used majority consensus to aggregate results, treating individual crowdworker results as votes.
To analyze agreement, Entropy \(E\) and normalized Herfindahl dispersion measure \(H^*\) were considered.
Agreement measures can be used to approximate result confidence and thus filter/flag crowd results with low agreement across crowdworkers.
A research challenge is to identify agreement measures that handle split votes\footnote{\(E\) and \(H^*\) rate a distribution like \((4,1,1,1,1)\) worse than \((4,4,0,0,0)\); both have \(f_e = 0.5\), but the agreement is relatively higher for the first distribution} better, e.g.~using Fleiss Kappa, and define an extended confidence-based quality control mechanism.

Visualizations as external representation of Knowledgebase \(\mathbb{K}_{B}\) as described in \cref{sec:extraction.platform} facilitate achieving the high level of code comprehension required for web migration.
AWSMAP provides different progress statistics as shown in \cref{fig:awsmap.statistics} and visualizations of concept relationships: Sankey diagrams as shown in \cref{fig:awsmap.sankey}, Force Graphs and Chord diagrams which represent the interlacing between artifacts and features or features and features, due to particular interest in this issue from medatixx.
Improvement of these visualizations requires further research including empirical studies on their usability and specification of better visualizations including also domain knowledge types in addition to features, integration of the visualizations into developer tools and specification of context-adaptable visualizations capable of handling high volumes of artifacts and knowledge in production-grade knowledgebases.

The rapid web migration prototyping technique of AWSM:RM is based on the emerging WebAssembly standard.
With Mozilla's ongoing development of the \emph{WebAssembly System Interface} (WASI)\footnote{\url{https://wasi.dev/}}, WASM is becoming an increasingly relevant technology no longer limited to the client side\footnote{cf.~\url{https://github.com/CraneStation/wasmtime/}}.
For the WASM-based infrastructure of AWSM:RM, support for POSIX-like system functions in the browser\footnote{currently available as polyfill: \url{https://wasi.dev/polyfill/}} can be used to increase the capabilities of ReWaMP prototypes and further reduce the manual code adaption effort, since fewer dependencies \(Dep\) of \(\mathfrak{L}\) will require expert changes due to higher availability of libraries and system functions.
To take advantage of the flourishing WASM environment, a review of the ReWaMP process and the RWMPA services is required once a first stable version of WASI becomes available.

For the rapid creation of grid-based web user interfaces from legacy non-web desktop user interfaces, the index approximations of \cref{eq:index-approximations} have been shown to produce reasonable quality with high performance in \cref{sec:uitransformer.experiment}.
The combined objective functions were rated lower and represent significant computational complexity.
To improve AWSM:RM, new objective functions and their combinations need to be investigated in order to provide a higher results quality.
At the same time, the problem of computational complexity must be further addressed, leveraging parallel computation and intermediate results sharing and the improvement potential through low-level programming languages demonstrated in \cref{sec:uitransformer.experiment}.

The visual UI similarity process presented in \cref{fig:ci-process} invites improvements in three parts.
The visual UI Element detection provides reasonable results for atomic UI Elements but for productive use, improved detection of composite elements and improved overall detection rate is required.
Our ongoing research thus focuses on enabling the use of deep neural networks, in particular using the Faster R-CNN architecture \autocite{Ren2017Faster-RCNN}, through synthesis of sufficiently large datasets and improving detection rate through application of the multi-agent system (MAS) paradigm to combine DOM-based with computer-vision-based detectors.
Object detection for UI Elements, however, is still a research challenge due to the distinct characteristics compared to object detection in general, as described in \cref{sec:segmentation.impl}, with applications beyond web migration.
The second potential improvement is the similarity approximation which is currently focused on Layout, but could be extended towards consideration of Material and integrated with existing works on Task and Behavior similarity.
The similarity measures will also benefit from new insights in perceptual UI similarity, which is still not well-understood in research.
The third relevant improvement opportunity is the provision of concrete guidance for the refinements based on the analysis results.
An improved solution could combine the measurement with detailed reporting and resolution advice similar to the USF platform for usability smells in web user interfaces \autocite{Grigera2017}.

\hypertarget{open-questions}{%
\subsection{Open Questions}\label{open-questions}}

This section briefly outlines research questions that have not been addressed and should be explored in future work.

As observed in our systematic mapping study \autocite{Heil2017Survey}, consideration of web migration for SME-sized ISVs in academia is very low.
While this thesis provided several contributions to address this gap, but there are still many research opportunities.
Specifically, the integration of web migration activities with ongoing agile development, which was shown to be the requirement with almost zero consideration in \cref{sec:sota.discussion}, poses a significant challenge.
The AWSM Methodology and Platform aimed for integration in the early phases of web migration and achieved a partial fulfillment of the requirement.
Thus, a challenging open research question is how to improve integration of web migration activities with ongoing agile development, not only for activities from migration initiation, but also for the migration phase such as target architecture specification, transformation or migration project monitoring.

Crowdsourcing has shown to be effective for the reverse engineering activity of concept assignment.
An open research question is the application of crowdsourcing in other reverse engineering areas.
The reformulation of concept assignment as classification problem, its assessment in the eight foundational dimensions of crowdsourcing \autocite{Latoza2016} and mapping to the microtasking model described in \cref{sec:csre} can serve as a blueprint strategy for the identification of suitable crowdsourcing models for other reverse engineering activities.

AWSM:RE with its OA-based ontological external knowledge representation is in line with recent efforts addressing standards-based data portability e.g.~for research data \autocite{Diaz2019OpenAnnotationInSLRs} and sharing platforms like hypothes.is\footnote{\url{https://hypothes.is}}.
An interesting research challenge is to investigate how reverse engineering using the W3C Web Annotation standard and OWL can benefit from the foundational knowledge sharing paradigm of LOD (Linked Open Data), the mature semantic web technology stack and the increasing research interest and active community in the context of the Solid\footnote{\url{https://solid.mit.edu/}} project.

WebAssembly at its current stage\footnote{our research on web migration prototyping started in early 2016, one year before WebAssembly reached cross-browser consensus and in 2017, to the best of our knowledge, we were the first to consider WASM in the general web migration context} was used as a technological enabler for Rapid Prototyping in the Web Migration domain based on the ideal of re-use and the available ISV staff expertise.
With the recent upswing of WASM and its increasingly rich environment - including package management\footnote{\url{https://wapm.io/}}, development environment\footnote{\url{https://webassembly.studio/}}, server-side runtimes\footnote{\url{https://wasmer.io/}} and integration with existing technologies like Microsoft's Razor Engine\footnote{\url{https://blazor.net/}} - the maturity level and acceptance of WASM as web technology brings up the research challenge of considering WASM beyond prototyping use cases as primary web migration target and explore resulting opportunities for re-use based web migration.

The successful transfer of paradigms from other domains into the software migration domain presented in this thesis, e.g.~Crowdsourcing from Software Engineering to Reverse Engineering and Rapid Prototyping from Agile Development and HCD to Web Migration, and in previous theses, e.g.~Knowledge Management \autocite{Razavian2013PHD} and Method Engineering \autocite{Khadka2016PHD} to SOA Migration, show the potential of reviewing web migration from the perspective of paradigms successful in other domains.
As observed in \cref{sec:approaches}, despite its success in Software Engineering, methods from Agile Development have hardly been considered for web migration with the exception of REMICS agile extensions \autocite{Krasteva2013REMICSAgile} and the integration of concept assignment into ongoing agile development outlined in \cref{sec:re.conceptual.integration}.
Thus an important research challenge is to investigate advances in web migration through transfer of knowledge from other domains (cf.~also to HCD's \emph{Analogous Inspiration} method \autocite{HCD2015}), in particular forward software engineering.

Understanding and analyzing the visual perception of user interfaces, their similarity and complexity is a relatively new research area the scope of which exceeds web migration and which enables various use cases including UI refactoring, usability and interaction quality prediction without extensive interaction tracing, targeted UI optimization, design search and design transfer \autocite{Bakaev2019JWE}.
The research challenge lies in the identification of appropriate objective, computable metrics that represent subjective visual perception of the user interface and to define methods and tools which make use of these insights to address the aforementioned use cases.
Our ongoing research in this field in the context of collaboration\footnote{\url{http://www.wuikb.online/en/platform/about}} with colleagues of NSTU Novosibirsk focuses on visual complexity \autocite{Bakaev2018APEIE}, application of Kansei Engineering for subjective similarity features \autocite{Bakaev2017Kansei}, metrics integration \autocite{Bakaev2019ICWE} and computer-vision based mining UI \autocite{Bakaev2018ICWE}.
The Aalto Interface Metrics (AIM) \autocite{Oulasvirta2018AIM} project\footnote{\url{https://interfacemetrics.aalto.fi}} follows a similar objective and agenda.
Understanding and analyzing visual perception of user interfaces is in its early stages and provides various research opportunities.

\textbf{END}

Todos:

\textbf{AWSM:RM}

AWSM:RM can address the intra-organisational resistance \autocite{Khadka2014ProfessionalsModernization,Sneed2010ReMiP} by concretising an abstract vision of a web-based version of the legacy system into a tangible software prototype which demonstrates feasibility and desirability thus helping to create culture that is more open towards this transformation process \autocite{Gartner2016Culture}.

NOT CONSIDERED: VIABILITY (HDC) \textgreater{} COST ESTIMATION ETC.
refs
