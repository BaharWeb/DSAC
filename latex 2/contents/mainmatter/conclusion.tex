\hypertarget{sec:conclusion}{%
\chapter{Conclusion and Outlook}\label{sec:conclusion}}

Concluding the thesis, this section summarizes the results and contributions, reflects on problems that were not addressed or require further investigation and points out open research challenges.

\vspace{-15pt}
\hypertarget{thesis-summary}{%
\section{Thesis Summary}\label{thesis-summary}}
\vspace{15pt}

Initially motivated by the observed difficulties of several \glspl{isv} to commence \gls{Web Migration} in the context of industrial research collaboration projects, this thesis systematically investigated the problem domain.
It outlined the reasons for \gls{Web Migration} on the one hand, and the complexity of the initial \glslink{Legacy System}{legacy} situation on the other hand, identifying effort and risk as the two main obstacles and deriving three corresponding research questions that define the scope of this thesis.
We detailed the situation of medatixx as representative sample of a capable modern medium-sized \gls{isv} struggling to bring its non-\glslink{web}{Web} \glspl{Legacy System} to the \gls{web}, analyzing characteristics of company, development, and \glspl{Legacy System} as well as migration objectives.
Based on a systematic field research and analysis process, requirements for an appropriate solution have been elicited, which detail the research questions and represent the problem analysis findings from the \gls{isv} stakeholder context.
Analysis of the state of the art based on a systematic mapping study revealed a lack of approaches supporting the initial phase of \gls{Web Migration} to address concerns about effort and risk that are tailored to the characteristics of \gls{sme}-sized \glspl{isv}, despite academia having shifted focus away from \gls{Web Migration}.
These shortcomings were further detailed as three main gaps in \gls{Web Migration} research.
To close these gaps, an \gls{hcd} ideation method was applied to create three specific research objectives under the overarching goal to support \gls{isv} in \gls{Web Migration} initiation with limited resources and expertise.

The proposed solution is \gls{awsm}, which provides a methodology for supporting \glspl{isv} to commence \gls{Web Migration}.
\gls{awsm} specifies principles, formalisms, methods, and tools which are complementary to existing comprehensive \gls{Web Migration} approaches and address their identified shortcomings.
The \gls{awsm} principles base it on open \gls{web} standards, shape it as a methodology for integration with \gls{Web Migration} approaches on different degrees of model-driven adoption, and advocate the use of \gls{Rapid Prototyping}.
The \gls{awsm} formalisms provide the conceptual basis ensuring interoperability through mathematical modeling and consistent mapping to the \gls{kdm} \gls{omg} standard.
The three \gls{awsm} Methods each address the gaps of existing approaches that contribute to \gls{isv}'s doubts about feasibility and desirability:

\begin{itemize}
\tightlist
\item
  AWSM:RE allows to identify and maintain existing valuable knowledge through crowdsourced \gls{Concept Assignment} supported by a \glslink{web}{Web}-based annotation platform.
\item
  AWSM:RM minimizes risk through migration pilots and demonstrates desirability and feasibility of a potential \glslink{web}{Web}-based version of the \gls{Legacy System} applying the \gls{Rapid Prototyping} paradigm to \gls{Web Migration}.
\item
  AWSM:CI allows to control the impact of \gls{Web Migration} on customers through measuring visible changes in the user interface.
\end{itemize}

The \gls{awsm} Toolsuite comprises tools that support these methods and the overall \gls{awsm} approach through automated or semi-automated \glspl{Transformation}, analysis, integration with existing development and management software, and providing queryable standards-based representations.
In particular, the \gls{awsm} Strategy Selection Decision Support System facilitates \gls{isv}'s specification of an overall \gls{Web Migration} strategy through faceted scenario-based search based on the data from our comprehensive systematic mapping study comprising 122 published \gls{Web Migration} approaches and software tools.

The \gls{awsm} Reverse Engineering Method achieves the research objective of identification and management of existing knowledge in legacy source code with limited resources and a lack of \gls{Web Engineering} expertise by facilitating recovery of problem and solution domain knowledge from the \glslink{Legacy System}{legacy} codebase for \glspl{isv} through a novel crowdsourced \gls{Concept Assignment} strategy.
AWSM:RE integrates with ongoing development as well as with other \gls{Web Migration} methods leveraging a queryable open \gls{web} standards-based knowledge representation.
The \gls{awsm} Risk Management Method achieves the research objective of demonstration of desirability and feasibility of a potential \glslink{web}{Web}-based version of the \gls{Legacy System} with limited resources and lack of \gls{Web Engineering} expertise by enabling the creation of a \glslink{web}{Web}-based version of the \gls{Legacy System} through a novel \gls{Rapid Web Migration Prototyping} strategy.
The \gls{awsm} Toolsuite supports this strategy with a WebAssembly-based toolchain and a guidance and automation system.
The \gls{awsm} Customer Impact method achieves the research objective of control of the impact of user interface changes through \gls{Web Migration} on customers with limited resources and lack of \gls{Web Engineering} expertise by enabling the measurement of visible change in user interfaces resulting from \gls{Web Migration} through a novel \gls{ui} Similarity measurement strategy.
The measurements can be calibrated to the target user group and are supported through a computer-vision based \gls{ui} Element detection and computation of \gls{ui} layout distances from visual analysis.

Effectiveness and applicability of the \gls{awsm} Methodology and Toolsuite has been evaluated in five different experiments combining empirical and objective data.
The experiments demonstrated the feasibility of the proposed techniques with limited resources and limited \gls{Web Engineering} and migration expertise, as well as the quality of the achievable results.
The support tools of the \gls{awsm} Toolsuite were consistently found useful by the test subjects.
The main problems identified under experimentation were task complexity, addressed by an additional guidance and automation system, and the non-mature nature of understanding perceived similarity of user interfaces, which is still a young field with ongoing research.
Analysis of the \gls{awsm} Methodology and Toolsuite in the context of stated requirements confirmed that the original objectives have been achieved.

\vspace{-15pt}
\hypertarget{lessons-learned}{%
\section{Lessons Learned}\label{lessons-learned}}
\vspace{15pt}

The research conducted for this PhD thesis yielded findings that are not specific to the \gls{awsm} Methodology and Toolsuite but relevant for research in the \gls{Web Migration} field and related areas in general:

\begin{itemize}
\tightlist
\item
  In academia, the interest in core \gls{Web Migration} topics has significantly dropped, with the majority of published approaches being of limited relevance due to outdated target \gls{web} technologies \autocite[cf. ][]{Heil2017Survey}.
Recently, the research focus shifted towards more modern technologies like Cloud.
However, many \glspl{isv} still struggle with basic \gls{Web Migration} and cannot target the Cloud.
The lack of research interest and resulting unavailability of solutions intensifies this problem.
\item
  A reconsideration of \gls{Web Migration} in the light of new technologies like WebAssembly can be fruitful, as they allow different migration paths due to opening up the formerly restricted technological environment of the \gls{web} towards various source technologies. \autocite[cf. ][]{Heil2018ReWaMP}
\item
  The transfer of paradigms that have been successful in forward software and \gls{Web Engineering} to \gls{Web Migration}, such as \gls{Crowdsourcing} \autocite{Heil2019CSRECCIS,Heil2018CSRE}, \gls{Rapid Prototyping} \autocite{Heil2018ReWaMP} and Visual \gls{ui} Analysis \autocite{Heil2016Similarity} can provide new impetus.
\item
  Despite fake contributions even in a limited group of crowdworkers, suitable quality control measures can effectively ensure result quality and some crowdworkers exhibit an unexpected degree of active commitment and investment of time to provide good result quality \autocite{Heil2019CSRECCIS,Heil2018CSRE}.
\item
  The effort for complex \gls{Web Migration} activities can be reduced through improved process guidance, partial automation, and heuristic suggestions, which is measurable in both required time and empirical evaluation (cf. \Cref{sec:rwmpa.experiment}).
\item
  Perceived similarity of user interfaces is important for \gls{Web Migration} and influence factors like Order, Orientation and Density constitute a very basic, applicable model \autocite{Bakaev2019JWE,Heil2016Similarity}, but research in this field is still very young and ongoing.
\item
  While a high degree of integration of \gls{Web Migration} activities with ongoing agile development is a crucial feasibility factor for \glspl{isv}, it has to be acknowledged that full integration is an ideal that cannot be achieved (cf. \Cref{sec:evaluation.stakeholder-requirements}).
As a requirement for \gls{Web Migration}, however, it is valuable since it has not been considered in research yet.
\end{itemize}

\hypertarget{sec:conclusion.contributions}{%
\section{Detailed Contributions}\label{sec:conclusion.contributions}}
\vspace{15pt}

This thesis produced models, methods, techniques, architectures, and tools that support \glspl{isv} to commence \gls{Web Migration}.
The following list provides a more detailed view of the contributions outlined in \cref{sec:introduction.contributions} by summarizing all research contributions of the thesis:

\begin{itemize}
\tightlist
\item
  Specification of a dedicated \gls{Web Migration} initiation methodology for \gls{sme}-sized \glspl{isv} with non-\glslink{web}{Web} \glspl{Legacy System} and large user bases.
\item
  Systematic problem analysis of the main stakeholder situation and identification of research objectives through a field-research-, \gls{hcd}- and LFA-based research process.
\item
  Systematic mapping study of both academic publications and software tools in the \gls{Web Migration} field with a focus on the \gls{sme} perspective.
\item
  Development of a Decision Support System facilitating \gls{Web Migration} strategy selection for \glspl{sme} through scenario-based, faceted search based on the systematic mapping data.
\item
  Definition of a role and process model for \gls{Reverse Engineering} through \gls{Concept Assignment} integrated into ongoing development and specification of a support toolsuite architecture integrated with \gls{isv}'s software infrastructure.
\item
  Definition of a queryable open \gls{web} standards-based representation of knowledge in \glslink{Legacy System}{legacy} codebases through ontological modeling and specification of a storage and querying \knowledgebase architecture.
\item
  Specification of a novel Crowdsourced \gls{Reverse Engineering} strategy through transfer of the \gls{Crowdsourcing} paradigm to the \gls{Reverse Engineering} domain by re-formulation of \gls{Concept Assignment} as classification problem.
\item
  Specification of a novel \gls{Rapid Web Migration Prototyping} strategy through transfer of the \gls{Rapid Prototyping} paradigm to the \gls{Web Migration} domain by leveraging current open \gls{web} standards.
\item
  Specification of a novel calibratable \gls{ui} similarity measurement strategy through modeling computable similarity measures by calculation of visual layout distances between non-\glslink{web}{Web} and \gls{web} user interfaces.
\item
  Insights on perceived similarity and its objective and subjective impact factors in the context of ongoing research on visual \gls{ui} similarity perception.
\item
  Systematic empirical experimentation with the proposed methods and tools combining both subjective and objective measures for analysis of effectiveness, applicability, and result quality.
\end{itemize}

\vspace{-15pt}
\hypertarget{ongoing-and-future-work}{%
\section{Ongoing and Future Work}\label{ongoing-and-future-work}}
\vspace{15pt}

The \gls{awsm} Methodology and Toolsuite addressed three crucial challenges related to initiation of \gls{Web Migration} for \glspl{isv}.
The proposed methods and tools aimed at lowering the initial barrier originating in effort and risk related to \gls{Web Migration} by dedicatedly addressing doubts about feasibility and desirability.
Legacy knowledge recovery, demonstration of feasibility, desirability and plausibility of a \glslink{web}{Web}-based version of the \gls{Legacy System} and customer impact control through \gls{ui} similarity measurements are enabled by various \gls{awsm} techniques.
The tools of the \gls{awsm} Toolsuite enable \glspl{isv} with limited resources and limited \gls{Web Engineering} and \gls{Web Migration} expertise to successfully apply these techniques.
Future work should focus on enhancing the efficiency and scope of the proposed mechanisms, explore other issues under the same motivation that are not addressed by \gls{awsm}, and further investigate the perception of user interfaces.

\hypertarget{methodology-and-platform-improvements}{%
\subsection{Methodology and Toolsuite Improvements}\label{methodology-and-platform-improvements}}
\vspace{10pt}

The \gls{awsm} Methodology and Toolsuite can be improved with regard to its functional scope and efficiency in several ways.
The \gls{awsm} Reverse Engineering Method based on \gls{Concept Assignment} specifies a generic and queryable representation of knowledge \(k=(t,r)\) (cf.~\cref{eq:knowledge}) and its location \(l\) in the codebase facilitated by the \gls{sckm} Ontology and SPARQL Endpoint.
Due to principle \cref{p:2} and \cref{p:3} no restrictions or further specifications are made for the internal representation \(r\).
Increasing the scope of AWSM:RE would allow putting more emphasis on the extraction process from extension \(l\) to intension \(k\), and a more detailed specification of \(r\).
This can be achieved in different ways: Combining AWSM:RE \gls{Concept Assignment} with automatic \gls{Reverse Engineering} methods can make use of the knowledge type information \(t\) to run dedicated knowledge extractors for the specific knowledge type on the related extensions, benefiting from AWSM:RE \gls{Concept Assignment} similar to the classifiers running on previously detected ROIs in \cref{sec:segmentation.impl}.
Integration with comprehensive \gls{Web Migration} approaches specifies \(r\) according to the specific models required, e.g.~UWA models, and adding model-specific extraction processes.
The third option is to extend our experiments on Crowdsourced \gls{Reverse Engineering} towards the extraction of specific problem and solution domain knowledge representations, e.g.~\gls{uml} diagrams of persistence models or \gls{bpmn} diagrams of business processes, by the crowd, for instance through microtasking with a more comprehensive classification ontology specific to \gls{Legacy System} \(\mathfrak{L}\).
This requires more research to transfer the benefits observed in CSRE and adapt the quality control measures to the new activity type.
%\todo{Integration similar to Codestream Slack Integration}
%https://www.golem.de/news/programmierung-codestream-ermoeglicht-diskussionen-am-code-per-slack-1908-142998.html

Within CSRE, balancing controlled disclosure with readability is a challenge.
\Cref{algocf:anonym} presents a simple anonymization technique, but AWSM:RE would benefit from a more sophisticated algorithm that improves crowdworkers' code comprehension while maintaining the three anonymization properties defined in \cref{sec:csre.anonym}.
For Crowdsourced \gls{Reverse Engineering}, quality control is crucial.
We used majority consensus to aggregate results, treating individual crowdworker results as votes.
To analyze agreement, Entropy \(E\) and normalized Herfindahl dispersion measure \(H^*\) were used.
Agreement measures can approximate result confidence and thus filter/flag crowd results with low agreement across crowdworkers.
A research challenge is to identify agreement measures that handle split votes\footnote{\(E\) and \(H^*\) rate a distribution like \((4,1,1,1,1)\) worse than \((4,4,0,0,0)\); both have \(f_e = 0.5\), but the agreement is relatively higher for the first distribution} better, e.g.~using Fleiss Kappa, and define an extended confidence-based quality control mechanism.

Visualizations as external representation of Knowledgebase \(\mathbb{K}_{B}\) as described in \cref{sec:extraction.platform} facilitate achieving the high level of code comprehension required for \gls{Web Migration}.
\gls{awsmap} provides different progress statistics as shown in \cref{fig:awsmap.statistics} and visualizations of concept relationships: Sankey diagrams, as shown in \cref{fig:awsmap.sankey}, Force Graphs and Chord diagrams, which represent the interlacing between \glspl{artifact} and features or features and features. %, due to particular interest in this issue from medatixx.
Improvement of these visualizations requires further research, including empirical studies on their usability and specification of better visualizations including also domain knowledge types in addition to features, integration of the visualizations into developer tools, and specification of context-adaptable visualizations capable of handling high volumes of \glspl{artifact} and knowledge in production-grade knowledge bases.

The \gls{Rapid Web Migration Prototyping} technique of AWSM:RM is based on the emerging WebAssembly standard.
With Mozilla's ongoing development of the \emph{WebAssembly System Interface} (WASI)\footnote{\url{https://wasi.dev/} Retrieved: 6.12.2019}, \gls{wasm} is becoming an increasingly relevant technology no longer limited to the client side\footnote{cf.~\url{https://github.com/CraneStation/wasmtime/} Retrieved: 6.12.2019}.
For the WASM-based infrastructure of AWSM:RM, support for POSIX-like system functions in the browser\footnote{currently available as polyfill: \url{https://wasi.dev/polyfill/} Retrieved: 6.12.2019} can be used to increase the capabilities of \gls{rewamp} prototypes and further reduce the manual code adaption effort, since fewer dependencies \(Dep\) of \gls{Legacy System} \(\mathfrak{L}\) will require expert changes due to higher availability of libraries/system functions.
To take advantage of the flourishing \gls{wasm} environment, a review of the \gls{rewamp} process and \gls{rwmpa} services is required once a first stable version of WASI becomes available.

For the rapid creation of grid-based \gls{web} user interfaces from \glslink{Legacy System}{legacy} non-\glslink{web}{Web} \glslink{Desktop Application}{desktop} user interfaces, the index approximations of \cref{eq:index-approximations} have been shown to produce reasonable quality with high performance in \cref{sec:uitransformer.experiment}.
The combined objective functions were rated lower and represent significant computational complexity.
To improve AWSM:RM, new objective functions and their combinations need to be investigated in order to provide a higher result quality.
At the same time, the problem of computational complexity must be further addressed, leveraging parallel computation and intermediate results sharing and the improvement potential through low-level programming languages demonstrated in \cref{sec:uitransformer.experiment}.

The visual \gls{ui} similarity process presented in \cref{fig:ci-process} invites improvements in three parts.
The visual \gls{ui} Element detection provides reasonable results for atomic \gls{ui} Elements, but for productive use, improved detection of composite elements and improved overall detection rate is required.
Our ongoing research thus focuses on enabling the use of deep neural networks, in particular using the Faster R-CNN architecture \autocite{Ren2017Faster-RCNN}, through synthesis of sufficiently large datasets and improving detection rate through application of the multi-agent system (MAS) paradigm to combine DOM-based with computer-vision-based detectors.
Object detection for \gls{ui} Elements, however, is still a research challenge due to the distinct characteristics compared to object detection in general, as described in \cref{sec:segmentation.impl}, with applications beyond \gls{Web Migration}.
The second potential improvement is the similarity approximation, which is currently focused on Layout, but could be extended towards consideration of Material and integrated with existing works on Task and Behavior similarity.
The similarity measures will also benefit from new insights in perceived \gls{ui} similarity, which is still not well-understood in research.
The third relevant improvement opportunity is the provision of concrete guidance for the refinements based on the analysis results.
An improved solution could combine the measurement with detailed reporting and resolution advice similar to the USF platform for usability smells in \gls{web} user interfaces \autocite{Grigera2017}.

\vspace{-10pt}
\hypertarget{open-questions}{%
\subsection{Open Questions}\label{open-questions}}
\vspace{10pt}

This section briefly outlines research questions that have not been addressed and should be explored in future work.

As observed in our systematic mapping study \autocite{Heil2017Survey}, consideration of \gls{Web Migration} for \gls{sme}-sized \glspl{isv} in academia is very low.
While this thesis provided several contributions to address this gap, but there are still many research opportunities.
Specifically, the integration of \gls{Web Migration} activities with ongoing agile development, which was shown to be the requirement with almost zero consideration in \cref{sec:sota.discussion}, poses a significant challenge.
The \gls{awsm} Methodology and Toolsuite aimed for integration in the early phases of \gls{Web Migration} and achieved a partial fulfillment of the requirement.
Thus, a challenging open research question is how to improve integration of \gls{Web Migration} activities with ongoing agile development, not only for activities from migration initiation but also for the migration phase such as target architecture specification, \gls{Transformation} or migration project monitoring.

\gls{Crowdsourcing} has shown to be effective for the \gls{Reverse Engineering} activity of \gls{Concept Assignment}.
An open research question is the application of \gls{Crowdsourcing} in other \gls{Reverse Engineering} areas.
The reformulation of \gls{Concept Assignment} as classification problem, its assessment in the eight foundational dimensions of \gls{Crowdsourcing} \autocite{Latoza2016}, and mapping to the microtasking model described in \cref{sec:csre} can serve as a blueprint strategy for the identification of suitable \gls{Crowdsourcing} models for other \gls{Reverse Engineering} activities.

AWSM:RE with its OA-based ontological external knowledge representation is in line with recent efforts addressing standards-based data portability e.g.~for research data \autocite{Diaz2019OpenAnnotationInSLRs} and sharing platforms like hypothes.is\footnote{\url{https://hypothes.is/} Retrieved: 6.12.2019}.
An interesting research challenge is to investigate how \gls{Reverse Engineering} using the \gls{w3c} Web Annotation standard and \gls{owl} can benefit from the foundational knowledge sharing paradigm of \gls{lod} (Linked Open Data), the mature semantic \gls{web} technology stack and the increasing research interest and active community in the context of the Solid\footnote{\url{https://solid.mit.edu/} Retrieved: 6.12.2019} project.

WebAssembly at its current stage\footnote{our research on web migration prototyping started in early 2016, one year before WebAssembly reached cross-browser consensus and in 2017, to the best of our knowledge, we were the first to consider \gls{wasm} in the general \gls{Web Migration} context} was used as a technological enabler for \gls{Rapid Prototyping} in the \gls{Web Migration} domain based on the ideal of reuse and the available \gls{isv} staff expertise.
With the recent upswing of \gls{wasm} and its increasingly rich environment -- including package management\footnote{\url{https://wapm.io/} Retrieved: 6.12.2019}, development environment\footnote{\url{https://webassembly.studio/} Retrieved: 6.12.2019}, server-side runtimes\footnote{\url{https://wasmer.io/} Retrieved: 6.12.2019} and integration with existing technologies like Microsoft's Razor Engine\footnote{\url{https://blazor.net/} Retrieved: 6.12.2019} -- the maturity level and acceptance of \gls{wasm} as \gls{web} technology brings up the research challenge of considering \gls{wasm} beyond \gls{Prototyping} use cases as primary \gls{Web Migration} target and explore resulting opportunities for reuse based \gls{Web Migration}.

The successful transfer of paradigms from other domains into the software migration domain presented in this thesis, e.g.~\gls{Crowdsourcing} from Software Engineering to \gls{Reverse Engineering} and \gls{Rapid Prototyping} from Agile Development and \gls{hcd} to \gls{Web Migration}, and in previous theses, e.g.~Knowledge Management \autocite{Razavian2013PHD} and Method Engineering \autocite{Khadka2016PHD} to \gls{soa} Migration, show the potential of reviewing \gls{Web Migration} from the perspective of paradigms successful in other domains.
As observed in \cref{sec:approaches}, despite its success in Software Engineering, methods from Agile Development have hardly been considered for \gls{Web Migration} with the exception of REMICS agile extensions \autocite{Krasteva2013REMICSAgile} and the integration of \gls{Concept Assignment} into ongoing agile development outlined in \cref{sec:re.conceptual.process}.
Thus, an important research challenge is to investigate advances in \gls{Web Migration} through transfer of knowledge from other domains, in particular forward software engineering.

Understanding and analyzing the visual perception of user interfaces, their similarity, and complexity is a relatively new research area the scope of which exceeds \gls{Web Migration} and which enables various use cases including \gls{ui} refactoring, usability and interaction quality prediction without extensive interaction tracing, targeted \gls{ui} optimization, design search and design transfer \autocite{Bakaev2019JWE}.
The research challenge lies in the identification of appropriate objective, computable metrics that represent subjective visual perception of the user interface and to define methods and tools which make use of these insights to address the aforementioned use cases.
Our ongoing research in this field in the context of collaboration\footnote{\url{http://www.wuikb.info/en/platform/about} Retrieved: 8.12.2019} with colleagues of NSTU Novosibirsk focuses on visual complexity \autocite{Bakaev2018APEIE}, application of Kansei Engineering for subjective similarity features \autocite{Bakaev2017Kansei}, metrics integration \autocite{Bakaev2019ICWE} and computer-vision based \gls{ui} mining \autocite{Bakaev2018ICWE}.
The Aalto Interface Metrics (AIM) \autocite{Oulasvirta2018AIM} project\footnote{\url{https://interfacemetrics.aalto.fi/} Retrieved: 6.12.2019} follows a similar objective and agenda.
Understanding and analyzing visual perception of user interfaces is in its early stages and provides various research opportunities.
%
%\subsubsection*{END}
%Todos:
%
%\subsubsection*{AWSM:RM}
%AWSM:RM can address the intra-organisational resistance \autocite{Khadka2014ProfessionalsModernization,Sneed2010ReMiP} by concretising an abstract vision of a web-based version of the \gls{Legacy System} into a tangible software prototype which demonstrates feasibility and desirability thus helping to create culture that is more open towards this \gls{Transformation} process \autocite{Gartner2016Culture}.
%
%NOT CONSIDERED: VIABILITY (HDC) \textgreater{} COST ESTIMATION ETC.
%refs
