\hypertarget{sec:requirements-analysis}{%
\chapter{Requirements Analysis}\label{sec:requirements-analysis}}

In this chapter, we will further investigate the problem raised in the introduction - concerns about effort and risk that render companies hesitant to commence a web migration due to insufficient support in initial phases - by outlining the research context in an industrial scenario and through the derivation of concrete requirements.

\hypertarget{sec:scenario}{%
\section{Scenario}\label{sec:scenario}}

The following scenario describes the external industrial context in which the research presented in this thesis was conducted.
It provides a concrete real-world example of the problems described in \cref{sec:problem}.
The scenario's characteristics with regard to company, software development, and source system provide the basis for the definition of requirements in \cref{sec:requirements} and the detailed problem analysis of the web migration approach elaborated in \cref{sec:solution}.

\hypertarget{sec:company-characteristics}{%
\subsection{Company and Software Development Characteristics}\label{sec:company-characteristics}}

The scenario is based on the situation of \emph{medatixx GmbH \& Co.~KG} at the beginning of the joint industrial research project ``eHealth Research Laboratory'' in October 2014.
Medatixx is an SME-sized \emph{independent software vendor}\footnote{https://www.gartner.com/it-glossary/isv-independent-software-vendor/} (ISV) of about 130 software developers plus 360 additional service and sales employees \autocite{Medatixx2018}.
Like most SME ISVs \autocite{Rose2016InnovationSME}, medatixx is successfully specialized in one particular sector, providing software solutions and IT Services for all kinds of resident doctors' offices and is the largest software provider in this sector in Germany with a market share of about 22\% \autocite{Medatixx2018}.
The core software development activities focus on information systems for patient management, so-called \emph{patient management software (PMS)}.
This type of software is subject to various regulatory constraints: general rulings like the GDPR to ensure data privacy \autocite{GDPR2016} as well as regulations specific to the medical sector like required certifications of PMS through the federal association of statutory health insurance physicians (KBV) \autocite{KBV2018}.
Updated regulations also impose a strict regime of quarterly release cycles that influences developer workload, maintenance activities, and evolution practices of the software products.
It is worth noting that the business processes in many doctors' offices are based on the processes represented in the PMS, making it hard to introduce more substantial changes in user interaction patterns or even replace the PMS.
As a result of several mergers and existing contracts with customers for maintenance and updates, the company continuously develops and evolves four similar main software products with overlapping features, different technologies and partially shared codebases.
Modernization activities are viewed in an in-house context; outsourcing is not desired.
The development staff is experienced in the technology bases and the functionality of the existing software products, but due to time and the mergers, the original developers and expertise are not entirely available anymore.
Web Engineering expertise is only limited; web migration expertise was not observed.
Incremental modernization of older software components is ongoing, focussing on re-development in C\#.NET.
A previous migration attempt towards a client/server architecture failed due to the complexity and lack of sufficient resources.

The software development activities are organized following agile practices and adoption of agile culture and mindset in work , and decision processes are at an advanced level for both software developers and the management of the development department.
The software developers are organized in teams of about 5-7 persons and integrate domain experts as testers.
These teams complete work from backlogs managed by product managers in a self-organized way.
Except for one team dedicated to corrective maintenance which adopts a Kanban \autocite{Anderson2010Kanban} process, the teams follow a Scrum-based \autocite{Beedle2002Scrum} iterative development model with minor variations across different teams.
Implementations of new features need to be approved by a group of UIX experts (Team U) to ensure consistently high usability.
Further agile methods and technologies such as physical sprint backlogs, time-framed efficient meetings with voluntary participation, continuous integration and automated software testing as well as knowledge sharing approaches like a gamification-driven wiki are present.
The above description characterizes Medatixx as a typical instance of the \emph{main stakeholder role} of this thesis: a capable modern SME-sized ISV with the problem of bringing its non-web legacy systems to the web.

\hypertarget{sec:scenario-code}{%
\subsection{Legacy Codebase Characteristics}\label{sec:scenario-code}}

For our joint research, we were kindly given access to the source code of one of the PMS called \emph{x.concept} by medatixx.
It provides functionality encompassing patient documentation and health records, managing appointments, prescriptions and billing.
The application is a traditional \emph{stand-alone desktop application with a graphical user interface (GUI)}, distributed via optical media and deployed via local installation on Windows-based personal computers in doctors' offices.
For distributed access from several workstations in different rooms, some doctors' offices are using MS Remote Desktop connections to a central PC that plays a server-like role.
To provide an overview of the general characteristics of the codebase, we used the source code analysis tool \emph{cloc}\footnote{https://github.com/AlDanial/cloc} version 1.80.
The source code consists of 34269 files, 30840 of which are unique (redundancy is mainly due to duplicate C/C++ Header files and some amount of code duplication), with an aggregated number of 8.8 million source lines of code (SLOC).
Source code analysis identified a heterogeneous landscape of 28 different programming languages and technologies, the vast majority of which (16.6k files, 6.5M SLOC) is Visual C++.
Other programming languages include C\# (7.2k files, 1.5M SLOC), C (244 files, 163k SLOC), and Pascal (150 files, 100k SLOC), technology-related artifacts include HTML (588 files, 149k SLOC), Windows Resource Files (767 files, 107k SLOC), MSBuild Scripts (437 files, 55k SLOC) and XAML (122 files, 17.8k SLOC).
The main technologies are Visual C++ as programming language, Microsoft Foundation Class (MFC)\footnote{http://msdn.microsoft.com/de-de/library/d06h2x6e.aspx} as GUI framework and FoxPro for persistence.
Further details can be found in \cref{tbl:zms-analysis}.
The software is structured in mostly isolated, in some cases even stand-alone components with several different communication mechanisms such as COM, IPC/Sockets, and MFC SendMessage.
While advanced metrics like functional size measurement (FSM) \autocite{ISO/IEC2009FSM} would provide more insight into the amount of functionality, automated calculation of function points is still not mature, and due to correlation with lines of code metrics \autocite{Albrecht1983FP} their improved objectivity has been questioned.
In any case, the size of 8.8M SLOC clearly qualifies x.concept as a large\footnote{only 12\% of the applications in the Appmarq repository - representing 1.03 billion LOC of 1850 applications - are larger than 1M LOC \autocite{CAST2017}} legacy system.

Within x.concept, \emph{ZMS} is an isolated stand-alone component which compiles into a separate executable handling the management of medical appointments.
It provides standard calendar functionality like day, 3-day, week and month calendar views, a task list, a vacation planner, a business hours schedule and management of patient appointments involving resources from medical staff and rooms.
Data flows for appointments go directly to an MS SQL Server via ODBC whereas resource data is loaded through Window-to-Window communication with the main application via MFC SendMessage.
ZMS serves as scenario application basis in this thesis.
\Cref{tbl:legacy_characteristics} shows abstract characteristics of the type of legacy software which we focus on in this thesis along with their instantiation in the ZMS scenario application.

\hypertarget{tbl:legacy_characteristics}{}
\begin{longtable}[]{@{}ll@{}}
\caption{\label{tbl:legacy_characteristics}Abstract Technical Characteristics of Legacy Software and Scenario Instantiation \autocite[adapted from][]{Heil2018ReWaMP}}\tabularnewline
\toprule
Abstract Characteristics & Instantiation in ZMS scenario application\tabularnewline
\midrule
\endfirsthead
\toprule
Abstract Characteristics & Instantiation in ZMS scenario application\tabularnewline
\midrule
\endhead
Legacy language & Visual C++\tabularnewline
CRUD functionality & CRUD for medical appointments\tabularnewline
Complex Business Logic & find next available time slot\tabularnewline
Desktop GUI & Microsoft Foundation Class (MFC)\tabularnewline
Component Communication & Window-to-Window via MFC SendMessage\tabularnewline
Third-party dependencies & DLLs via assembly loading\tabularnewline
File and Data Base Persistence & Files and MS SQL Server via ODBC\tabularnewline
\bottomrule
\end{longtable}

\hypertarget{migration-objectives}{%
\subsection{Migration Objectives}\label{migration-objectives}}

Current PMS like x.concept primarily support doctors and nurses.
Future healthcare applications, however, should support all roles involved in ambulant healthcare, such as patients, pharmacists or transport providers.
The situation is comparable to the banking sector twenty years ago when banking information systems only supported bank clerks as actors.
With the increasing ubiquity of web-based systems, the extension of these formerly closed systems empowering customers as system actors has lead to the successful establishment of online banking as the de-facto standard for internet users \autocite{BitkomResearch2016DigitalBanking} and enabled the creation of new business models creating the FinTech sector \autocite{Schueffel2016FinTech}.
Similar effects have been observed for the travel sector.
Empowering the physicians' customers, i.e.~the patients, in the context of the ZMS scenario application means to provide them with functionality to book their medical appointments based on availability and personal preferences, to re-schedule or cancel appointments, to receive reminders and delay notifications and to access this information independent of the doctors' business hours and integrate the appointments in calendar applications on their personal mobile devices.
Comparable functionality is familiar for end users, e.g.~in the context of flight bookings.
Providing this increased functionality and comfort to patients - the customers of the company's customers - generates an added value and a potential competitive advantage for doctors - the customers of the company - and therefore is desirable for the PMS of the ISV.

To enable this, however, a distributed, web-based system is required, using open Web standards to provide a high degree of interoperability.
The current locally installed, closed PMS cannot deliver the required interactivity, because its installations run on PCs in the doctors' offices - often only during business hours - and cannot be accessed or interacted with from outside.
Integration of further third-party roles like pharmacists and transport providers for federated scenarios would require standardized communication interfaces and authorization mechanisms.
The required web migration poses a challenge for the company and initiating this process is difficult as outlined in \cref{sec:problem} and further analyzed in \cref{sec:problem-analysis-results}.

\hypertarget{sec:requirements}{%
\section{Requirements}\label{sec:requirements}}

In this section, requirements for assessing the state of the art are elicited based on the situation and motivation outlined in the introduction and the scenario.
These requirements are divided into two groups:

\begin{itemize}
\tightlist
\item
  \textbf{Scope requirements} are detailing criteria that ensure that a solution addresses the scope set by the main research question RQ1.
\item
  \textbf{Stakeholder requirements} are detailing criteria that ensure the appropriateness of a solution for an ISV as detailed in the scenario, based on existing literature and analysis of the scenario.
\end{itemize}

To indicate different requirements levels, we follow the definitions of RFC 2119 \autocite{IETF2119MustShould}, using

\begin{itemize}
\tightlist
\item
  \textbf{MUST} for absolute requirements fulfillment of which is mandatory, and
\item
  \textbf{SHOULD} for requirements fulfillment of which is recommended.
\end{itemize}

The following table provides an overview of all scope and stakeholder requirements and indicates their significance according to RFC 2119 levels.

\begin{longtable}[]{@{}lll@{}}
\toprule
ID & Requirement & Level\tabularnewline
\midrule
\endhead
\textbf{Scope Requirements} & &\tabularnewline
S1 Initial & Initial Phase Support & MUST\tabularnewline
S2 Web & Web Application Target & MUST\tabularnewline
\textbf{Stakeholder Requirements} & &\tabularnewline
C1 Risk & Risk Management & SHOULD\tabularnewline
C2 Reuse & Re-use of Legacy Assets & SHOULD\tabularnewline
C3 Exp & Expertise \& Tool Support & SHOULD\tabularnewline
C4 Agile & Agile Development Process Integration & SHOULD\tabularnewline
\bottomrule
\end{longtable}

\hypertarget{scope-requirements}{%
\subsection{Scope Requirements}\label{scope-requirements}}

The following two requirements detail the scope of this thesis to consider web migration approaches that support the initial phase of a web migration and which are designed to produce web applications from legacy systems.

\textbf{S1 Initial Phase Support}

Companies find it particularly difficult to commence a web migration \autocite{Heil2017Survey,Heil2018ReWaMP}.
This requirement, therefore, assesses approaches with regard to how much they support a software developing company to overcome the initial resistance.
According to the \emph{Reference Migration Process (ReMiP)} \autocite{Sneed2010ReMiP}, any software migration process can be divided into four phases: Preliminary Study, Conceptualization and Design, Migration and Transition and Closing down.
A similar phase distinction is also found in EU FP7 REMICS \autocite{Krasteva2013REMICSAgile}: requirements and feasibility, recover, migrate, validation.
The first two phases in both reference models comprise the initial activities before the actual transformation is started in the third phase, with activities including evaluation of the legacy system, legacy analysis and definition of migration strategy and target system.
Technically required for any reengineering or transformation approach, recovery of knowledge plays a special role in migration and is therefore considered separately in \emph{C2 Reuse}.
Thus, approaches covering activities from the first two phases that are not related to recovery are considered to support the initial phase to some extent.
However, the mainly technical perspective of the reference models' phases ignores the importance of communication of the necessity and benefits of modernization - communicating the business value is crucial for creating a \emph{business case}\footnote{``a documented economic feasibility study used to establish validity of the benefits of a selected component lacking sufficient definition and that is used as a basis for the authorization of further project management activities'' \autocite{ISO/IEEE24765Vocabulary}} for migration \autocite{AmazonWebServices2018Migration,Khadka2016PHD,Menychtas2014ARTISTJournal,Batlajery2014IndustrialSurveyModernization,Khadka2014ProfessionalsModernization} that justifies the project cost with benefits \autocite{Sneed1995CostBenefit} - to address migration resistance within organizations \autocite{Khadka2014ProfessionalsModernization,Sneed2010ReMiP}.
A suitable approach should therefore additionally provide appropriate methods and tools to supply artifacts which can be used as a basis for communicating the necessity and benefits of migration and supporting the decision making.

The satisfaction criterion of this requirement is the support for both the technical and the communication perspective of the initial phase, which means that activities prior to the actual migration are addressed and include support for communicating necessity and benefits.
The requirement is partially satisfied if activities prior to actual migration are addressed, but communication is ignored.
Finally, the requirement is considered not satisfied if only activities from the third and fourth ReMiP/REMICS phase are addressed.

\textbf{S2 Web Application Target (Target Environment)}

Due to the advantages of the web platform \autocite{Gitzel2007WebEngineeringMDD,Knorr2003WebAsPlatform} and to meet ISVs' migration objectives with regard to the integration of roles and interactivity as described in \cref{sec:situation}, a web-based system is required.
\emph{Software migration} is the process of moving an existing software system from one environment to another \autocite{SWEBOK2014}.
Generic software migration approaches, however, do not sufficiently address the web paradigms, like the asynchronous request-response communication model, client-server separation in the spatial and technological dimension or URL-based resources and addressable UI states/navigation patterns \autocite{Heil2017Survey}.
In the context of web migration, the \emph{target environment} is the web platform.
Web migration approaches comprise different types of target web systems \autocite[cf.~][]{def:websystem} such as Web Applications, SOA-based and Cloud-based systems \autocite{Heil2017Survey}.
Only a subset of these approaches explicitly focuses on Web Applications as defined in \autocite{def:webapplication}.
Since the migration objectives require a web application including a web-based user interface, approaches for SOA or Cloud migration are only partially applicable.
It is important to note that according to \autocite{def:webapplication}, not every Web System with a web-based UI is a Web Application since in addition provision of web-specific resources is required.

The satisfaction criterion of this requirement is the Web Application nature of the software that is produced by an approach, which means that the resulting software system is based on web technologies and standards and provides Web-specific resources through a web-based user interface.
The requirement is partially satisfied if the software is a Web System, but lacks the aspect of a web-based user interface.
Finally, the requirement is considered not satisfied if an approach only addresses the development or migration of software in general without consideration of the Web as target environment.

\hypertarget{stakeholder-requirements}{%
\subsection{Stakeholder Requirements}\label{stakeholder-requirements}}

The following four requirements detail concerns about commencing a web migration from an ISV's perspective by further detailing risk and effort.

\textbf{C1 Risk Management}

As outlined \cref{sec:problem}, any software modernization involves a high risk, which makes companies hesitant to commence this process \autocite{Khadka2014ProfessionalsModernization,Canfora2000Decomposing,Bisbal1999LegacyInformationSystems,Heil2018ReWaMP}.
The risks can be divided into two categories: \emph{risks of migration process failure} and \emph{risks of failure of the resulting new system}.
Both categories require appropriate strategies to manage and decrease the risk level.
Redevelopment approaches from scratch without consideration of the existing software system and artifacts, called ``Cold Turkey'' \autocite{Brodie1995Migrating} and holistic cut over strategies, called ``Big Bang'' \autocite{Bisbal1999LegacyInformationSystems}, have a high risk of failure \autocite{Sneed2010SoftwareMigration,Bisbal1999LegacyInformationSystems}.
Typical \emph{risk management} \autocite{ISO/IEEE24765Vocabulary} and mitigation strategies for large migration projects are to follow an incremental migration process \autocite{Colosimo2007ControlledExperiments,Sneed2010SoftwareMigration} - originally introduced as package-oriented incremental ``\emph{chicken little}'' strategy \autocite{Brodie1995Migrating} - and to use pilot projects as trial migrations to identify obstacles such as feasibility threats early on \autocite{AmazonWebServices2018Migration,Sneed2010SoftwareMigration}.
\emph{Feasibility studies} assessing the technical feasibility and economic viability \autocite{ISO/IEEE24765Vocabulary} are another means of basic risk management commonly observed.
\emph{Portfolio analysis} is a risk management method to identify potential migration candidates using a quadrant graph of business value and technical quality \autocite{Seacord2003ModernizingLS,Sneed1995CostBenefit}.
For these candidates, stakeholders and requirements are identified to finally make the \emph{business case}, a document to support decision making and planning \autocite{AmazonWebServices2018Migration,Seacord2003ModernizingLS}.
Advanced risk management methods should thus contribute to the business case \autocite{Seacord2003ModernizingLS}.
Any modernization bears the risk of losing valuable tacit knowledge about business processes, rules, etc.
\autocite{Aversano2001,Distante2006a,Sneed2010SoftwareMigration,Wagner2014Fundamentals} which are not explicitly documented but only implicitly represented by the legacy source code \autocite{Khadka2014ProfessionalsModernization,Heil2017Survey}.
This knowledge represents high financial value, resulting from high investments in the past \autocite{Lucia2009METAMORPHOS}.
A \emph{risk-managed modernization} strategy \autocite{Seacord2003ModernizingLS} therefore should provide methods to re-discover, secure and manage this knowledge to make it useable for subsequent modernization activities.

The satisfaction criterion of this requirement is the risk management of the web migration, which means that at least one basic risk management practice beyond incremental process is combined with advanced strategies contributing to the business case and the securing of existing knowledge against loss during migration.
The requirement is partially satisfied if the approach only employs basic risk management practices beyond incremental process model, but lacks contribution to the business case and the securing of knowledge.
Finally, the requirement is considered not satisfied if an approach does not take into account risk management.

\textbf{C2 Re-use of legacy assets}

\todo{TODO:RENAME\_continuity\_of\_functionality\_and\_UIX?}

Redevelopment from scratch - sometimes also referred to under the category of \emph{replacement} \autocite{Almonaies2010SOAStrategies} - is a major strategy for legacy modernization \autocite{Wagner2014Fundamentals,Khadka2016PHD,Sneed2010SoftwareMigration,Almonaies2010SOAStrategies,Bisbal1999LegacyInformationSystems}.
However, its lack or limited reuse of existing legacy assets incurs significant development cost \autocite{Khadka2016PHD} and bears the risk of the new system not being as functional as the old one \autocite{Almonaies2010SOAStrategies}.
Maintaining a system's functionality throughout any software modernization means to maintain the domain knowledge and business logic \autocite{Wagner2014Fundamentals}.
From an end user's perspective, it also means continuity in the user interface and user interaction.
Often, companies aim at maintaining the look and feel of the legacy user interface to avoid forcing end-users to change their working habits \autocite{Rodriguez-Echeverria2012MIGRARIA,Lucia2008,Distante2002}.
Existing \emph{legacy artifacts}\footnote{In this thesis we distinguish between software artifacts and assets.
``A software artifact is a tangible machine-readable document created during software development.
Examples are requirement specification documents, design documents, source code, and executables.'' \autocite{OMG2016KDM}, \autocite[cf.~\emph{physical asset}][]{ISO/IEEE24765Vocabulary}} of an ISV comprise the legacy source code and the running legacy system itself, whereas \emph{assets}\footnote{An asset is an ``item, thing or entity that has potential or actual value to an organization'' \autocite{ISO/IEEE24765Vocabulary}.
``A software asset is a description of a partial solution \ldots{} or knowledge \ldots{} that engineers use to build or modify software products.'' \autocite{OMG2016KDM} Parts of software can be an artifact and an asset at the same time (e.g.~documentation).
Unless explicitly codified, we consider requirements, models, rules, etc.
represented by the legacy source code as assets but not artifacts, since they are not tangible \autocite[cf.~\emph{intangible assets}][]{ISO/IEEE24765Vocabulary}.} like documentation, requirements or models originally used for production are typically missing.
\autocite{Wagner2014,Bisbal1999LegacyInformationSystems,Sneed2010SoftwareMigration,warren2012renaissance,Batlajery2014IndustrialSurveyModernization,Lucia2008}.
Thus, original requirements and models need to be rediscovered from existing software artifacts to enable reuse, employing appropriate reverse engineering techniques such as static analysis of legacy source code or dynamic analysis of system behavior \autocite{Lucia2008}.
A suitable approach must promote rediscovery, management, and reuse of assets in existing software artifacts to increase the continuity of functionality \autocite{Almonaies2010SOAStrategies} and user interaction between legacy and new system, in order to reduce the development cost \autocite{Khadka2016PHD} and accelerate modernization compared to redevelopment from scratch \autocite{Sneed2010SoftwareMigration}.

The satisfaction criterion of this requirement is the degree of re-use of assets from existing software artifacts, which means that legacy source code and the running legacy system itself should be used as the source for maintaining functionality and user interaction across legacy and new system.
The requirement is partially satisfied if either only functionality or only user interaction is maintained.
Finally, the requirement is considered not satisfied if existing assets are not taken into account for the continuity of functionality or user interaction.

\textbf{C3 Expertise \& Tool Support}

\emph{Expertise} of the available staff is a crucial resource for any software modernization \autocite{Khadka2014ProfessionalsModernization,Batlajery2014IndustrialSurveyModernization,Sneed2010SoftwareMigration,Seacord2003ModernizingLS}.
While expertise in the source technology base in companies with continuous maintenance and development activities due to existing contracts and strict release cycles as described in \cref{sec:company-characteristics} is typically high, experience with older parts of the legacy system can be significantly lower due to retirement or change of job of the original developers \autocite{Khadka2016PHD,Batlajery2014IndustrialSurveyModernization,Khadka2014ProfessionalsModernization}, called \emph{erosion of soft knowledge} \autocite{Khadka2014ProfessionalsModernization}.
Lack of staff experienced in software modernization itself is a feasibility threat \autocite{Sneed2010SoftwareMigration,Seacord2003ModernizingLS}.
For web migration, additional web engineering expertise is required due to the specifics of the target environment (cf.~\emph{S2 Web}), but typically not existing in ISVs of traditional non-web desktop applications \autocite{Fowley2017CloudSME}.
For a company to be able to commence a web migration with existing staff and expertise, approaches need to consider the four experience levels explained above in terms of its requirements and provide supporting infrastructure for partial or full \emph{automation} and proper \emph{guidance} in the migration process.
The effort for reverse engineering and migration activities can be significantly reduced by suitable support tools \autocite{Lucia2009METAMORPHOS}, which only see limited use in the industry \autocite{Torchiano2008ItalianSurvey}.
Additional staff or outsourcing\footnote{cf.~also to \autocite{Razavian2012} for strategy differences between in-house and outsourced migration projects} the entire migration is not desirable due to limited financial resources and complexity of the legacy system respectively {[}cf.~\#sec:company-characteristics{]}.

The satisfaction criterion of this requirement is the expertise requirements, which means that suitable approaches are feasible with existing staff based on experience levels in legacy technology, legacy system, target technology, and migration, supported, if necessary, through appropriate tools and proper guidance in the migration process.
The requirement is partially satisfied if the approach only meets the staff expertise, but lacks tool support.
Finally, the requirement is considered not satisfied if additional staff or outsourcing is required.

\textbf{C4 Agile Development Process Integration}

Software providers have \emph{ongoing development and maintenance activities} as described in \cref{sec:company-characteristics}.
Organization structures are representing these activities with teams' responsibilities assigned to maintenance of products or components and implementation of new features, following \emph{agile development methodologies} that govern work organization in terms of process, roles and artifacts \autocite{Beedle2002Scrum}.
The importance of supporting existing ways of working has only recently been acknowledged in lean perspectives on migration research \autocite{Razavian2014a}.
Additional activities required by web migration thus need to be integrated with day-to-day development activities of existing teams to avoid re-structuring or even exclusively dedicating teams to migration \autocites[cf.~\emph{modernization teams} in][]{Krasteva2013REMICSAgile}[\emph{migration factory teams} in][]{AmazonWebServices2018Migration}.
Of the seven disciplines according to ReMiP \autocite{Sneed2010ReMiP,Gipp2007ReMiP}, target design, implementation, test, and deployment are very similar to traditional forward software engineering and therefore easier to integrate.
In contrast, requirements analysis - for migration this means eliciting un-documented requirements from legacy artifacts, mainly the legacy code-, legacy analysis, and strategy selection are migration specific activities that are significantly different from regular software engineering activities and therefore more difficult to integrate.
Integration is not only required at the process level in terms of activities, but also for artifacts that are created and drive the development.

The satisfaction criterion of this requirement is the completeness of development integration, which means that all migration activities are integrated into ongoing development processes and integration is also achieved at the artifacts level.
The requirement is partially satisfied if not all activities are integrated, or integration of artifacts is lacking.
Finally, the requirement is considered not satisfied if an approach does not integrate with ongoing development and is designed to be stand-alone.

\hypertarget{summary}{%
\section{Summary}\label{summary}}

This chapter introduced a guiding scenario of an SME-sized ISV struggling to bring its non-web legacy systems to the web and detailed the characteristics of the company, software development, and legacy system, and described migration objectives.
Six requirements were elicited, the two scope requirements represent the problem identified as scope of this thesis in RQ1, i.e., support for the initial phases of web migration and web applications as target architecture, and the four stakeholder requirements are derived from the problems of the scenario stakeholder such as management of risk, reuse of legacy assets to ensure continuity of functionality and user interaction, suitability for available expertise and tool support, and integration into ongoing agile development and maintenance activities.
For each of the requirements, a three-level assessment scheme was introduced enabling the evaluation of existing web migration approaches regarding their suitability in the following chapter.
